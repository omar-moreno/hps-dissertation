The Heavy Photon Search (HPS) is a new experiment at Jefferson Lab that will 
search for heavy $U(1)$ vector bosons (heavy photons, dark photons or $A'$)
in the mass range of 20 MeV/c$^{2}$ to 500 MeV/c$^{2}$ that couple weakly to
ordinary matter.  Heavy photons in this mass range are theoretically favorable
and may also mediate dark matter interactions.  The heavy photon couples to 
electric charge through kinetic mixing with the photon, in turn, inducing an 
effective gauge coupling of the $A'$ to electric charge, which is suppressed
relative to the electron charge by a factor of 
$\epsilon \sim 10^{-2} - 10^{-12}$.  Since heavy photons couple to electrons, 
they can be produced through a process analogous to bremsstrahlung radiation, 
subsequently decaying to narrow $e^{-}e^{+}$ resonances which can be observed 
above the dominant QED trident background.  For suitably small couplings, dark
photons travel detectable distances before decaying, providing a second 
signature.

HPS will utilize this production mechanism to probe heavy photons with relative
couplings of $10^{-5} - 10^{-10}$ and search for the $e^{-}e^{+}$ decay of the
heavy photon via two signatures: invariant mass and displaced vertex.  Using 
Jefferson Lab’s high luminosity electron beam along with a compact, large 
acceptance forward spectrometer consisting of a silicon vertex tracker and lead
tungsten electromagnetic calorimeter, HPS will access unexplored regions in the
mass/coupling space. 

The HPS engineering run took place in spring of 2015 using a 1.1 GeV, 50 nA 
beam.  This thesis will present the results of a resonance search for a heavy
photon in the mass range between 20 MeV/c2 to 100 MeV/c2 using the engineering
run data. 

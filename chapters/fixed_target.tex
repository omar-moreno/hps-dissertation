
\chapter{Fixed Target Signal and Backgrounds}

\section{Heavy Photon Production}

Sensitivity to the theoretically favored regions of the heavy photon 
mass-coupling phase space can be best achieved using high luminosity fixed
target experiments \cite{PhysRevD.80.075018}.  In such experiments, an electron
of energy $E_{0}$ incident on a high $Z$ target will radiate heavy photons 
through a process analogous to ordinary photon bremsstrahlung.  The differential
(Fig. \ref{fig:ap_production}).  
\begin{figure}[t]
    \centering
    \caption{A' being produced.}
%   \includegraphics[width=0.5\textwidth]{images/}
    \label{fig:ap_production}
\end{figure}  
cross-section of such a process can be estimated using the Weizacker-Williams 
approximation as 
\[
    \frac{\sigma}{dx d\cos{\theta_{A}}} \approx \frac{8 Z^{2} \alpha^{3} \epsilon^{2} E_{0}x}{U^{2}} \chi
            \time [(1 - x + x^{2}/2) - x(1-x)m_{A'}^{2}E_{0}^{2}x\theta_{A'}^{2}/U^{2}]
\]
where $\alpha \sim 1/137$ is the fine structure constant, $\theta_{A}$ is the 
scattering angle of the $A'$, $\chi$ is the effective photon flux and 
\[
    U(x, \theta_{A'}) = E_{0}^{2}x\theta_{A'}^{2} + m_{A'}^{2}\frac{1-x}{x} + m_{e}^2 x
\]
is the virtuality of the intermidiate electron.

%In the case that the mass of the heavy photon is zero, equation
Although heavy photons are produced in a process similar to ordinary bremsstrahlung, 
their production rate and kinematics differ in several ways: 
\begin{itemize}
    \item The production cross-section is suppressed by a factor of $\epsilon^{2}m_{e}^{2}/m_{A'}^{2}$.
    \item The $A'$ is produced very forward.
    \item The $A'$ will take most of the incident beam energy.
\end{itemize}

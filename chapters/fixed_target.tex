
\chapter{HPS Signal and Backgrounds}

The Heavy Photon Search is a fixed target experiment that will search for heavy
photons in the mass range of 20 MeV/c$^{2}$ to 500 MeV/c$^{2}$ and couplings
of $\epsilon \sim 10^{-5} - 10^{-10}$.  Since heavy photons couple to electric
charge, they can be produced by a process analogous to bremsstrahlung 
radiation.  The heavy photon subsequently decays to narrow $e^+e^-$ resonances, 
which can be observed above the dominat quantum electrodynamic (QED) trident
background.  For suitably small couplings, heavy photons travel detectable 
distances before decaying providing a second signature.  In the chapter that
follows, both the heavy photon production mechanism and backgrounds will be 
discussed.

\section{Production of Heavy Photons}

Sensitivity to the theoretically favored regions of the heavy photon 
mass-coupling phase space can be best achieved using high luminosity fixed
target experiments \cite{PhysRevD.80.075018}.  In such experiments, an electron
of energy $E_{0}$ incident on a high $Z$ target will radiate heavy photons 
through a process analogous to ordinary photon bremsstrahlung 
%
% Do I need to explain what bremsstrahlung is or should I assume it's common
% knowledge?
%
(Fig. \ref{fig:ap_production}).  However, as discussed below, the weak coupling
of the $A'$ to electrons along with it's relatively larger mass will lead to 
rates and kinematics which are very different from ordinary photon 
bremsstrahlung.

The energy-angle distribution of heavy photons in such a process 
%%%%%%%%%%%%%%
%   Figure   %
%%%%%%%%%%%%%%
\begin{figure}[t]
    \centering
    \caption{A heavy photon can be produced through a process analogous to 
             ordinary photon bremsstrahlung.}
    \includegraphics[width=0.5\textwidth]{images/aprime_brem.png}
    \label{fig:ap_production}
\end{figure}  
%%%%%%%%%%%%%%
%%%%%%%%%%%%%%
process can be estimated using the Weizacker-Williams approximation as 
\begin{equation}
    \label{eqn:ap_diff_cross}
    \frac{d\sigma}{dx d\cos\theta_{A'}} =
    \frac{8 \alpha^{3} \epsilon^{2} E_{0}^2 x \sqrt{1-m_{A'}^{2}/E_{0}^{2}}}{U^{2}} \chi
    \left [ \left (1 - x + \frac{x^{2}}{2} \right ) 
    - \frac{(1-x)^{2} m_{A'}^{2}}{U^{2}}
    \left(m_{A'}^{2} - \frac{Ux}{1-x} \right) \right]
\end{equation}
where $\alpha \sim 1/137$ is the fine structure constant, $\theta_{A}$ is the
opening angle of the $A'$ relative to the incident electron in the lab frame, 
$x = E_{A'}/E_{0}$, $\chi$ is the effective photon flux and 
\begin{equation}
    U(x, \theta_{A'}) = E_{0}^{2}x\theta_{A'}^{2} 
    + m_{A'}^{2}\frac{1-x}{x} + m_{e}^2 x
\end{equation}
is related to the virtuality of the itermidiate electron \cite{PhysRevD.34.1326,
PhysRevD.80.075018}.  Assuming
$m_{e} << m_{A'}$, 
and integrating \ref{eqn:ap_diff_cross} over all angles yields
\begin{equation}
    \label{eqn:ap_diff_cross_i}
    \frac{d\sigma}{dx} = \frac{8\alpha^{3}\epsilon^{2} \sqrt{1-m_{A'}^{2}/E_{0}^{2}}}
    {m_{A'}^{2}\frac{1-x}{x} + m_{e}^{2}x}\chi
    \left( 1 - x + \frac{x^{2}}{3}\right).
\end{equation}
Although \cite{eqn:ap_diff_cross_i} reduces to the cross-section of photon 
bremsstrahlung in the limit that $m_{A'} \rightarrow 0$, their production rate
and kinematics differ in several ways. Specifically, the production 
cross-section is suppressed by a factor of $\epsilon^{2}m_{e}^{2}/m_{A'}^{2}$.
Furthermore, the $A'$ is produced very forward and will take most of the 
incident beam energy. \textbf{Expand this a bit more ... Also talk about the
total rates.}

\section{Trident Backgrounds}

The primary background expected to dominate the final event sample of the Heavy
Photon Search experiment is the quantum electrodynamic Trident process.  As 
shown on Fig. (\cite{fig:tridents}), the tridents can be seperated out into two
main diagrams: 
\begin{figure}[t]
    \begin{subfigure}{.5\textwidth}
        \centering
        \includegraphics[width=0.8\textwidth]{images/bethe-heitler.png}
    \end{subfigure}
    \begin{subfigure}{.5\textwidth}
        \centering
        \includegraphics[width=0.8\textwidth]{images/radiative.png}
    \end{subfigure}
    \caption{Diagrams of the radiative and Bethe-Heitler trident reactions.}
    \label{fig:tridents}
\end{figure}  
Bethe-Heitler and radiatives. The kinematics of radiativesare indistinguishable
from the $A'$ signal events within an invariant mass window near the $A'$ mass.
Therefore, radiatives can be used to analyze both the rate of the $A'$ signal 
production and the sensitivity of an experiment to $A'$ signals.  Specifically,
the $A'$ production cross-section is related to the production cross-section of 
radiatives as 
\[
    \frac{d\sigma(A')}{d\sigma(\gamma^*)} = \frac{3\pi\epsilon^{2}}{2 N_{eff} \alpha}
        \frac{m_{A'}}{\delta m}
\]
As a result, the radiatives in the final event sample can be used to analyze 
the $A'$.

Although the rate of the Bethe-Heitler process dominates among the two processes, 
its different kinematics can be used to reduce them in the final event sample.
Specifically, the $A'$ decay products are highly boosted while the recoiling 
electron is soft and scatters at large angles.  In contrast, at higher 
energies, the Bethe-Heitler process is not enhance.  Furthermore, only one of
the leptons in the pair will be highly boosted, while the other will be much
softer.  The recoiling electron will be produced much more forward.



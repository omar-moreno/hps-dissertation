
\chapter{The HPS Apparatus}

%The HPS experiment engineering run was conducted in the Spring of 2015 at the
%Thomas Jefferson National Accelerator Facility (JLab) in Newport News, VA.  The
%HPS detector was installed within the Hall B alcove upstream of the Continuous 
%Electron Beam Accelerator Facility (CEBAF) Large Acceptance Spectrometer for 12
%GeV (CLAS12) detector.  HPS utilized CEBAF's high luminosity electron beam,
%operating at an energy of 1.056 GeV and current of 50 nA, incident on a thin
%(~0.125\% $X_{0}$) tungsten target to search for an $A'$ with a mass in the 
%range of 20 - 100 MeV.  

At the energies at which the HPS experiment is operating, the 
electroproduced $A'$ will carry most of the incident beam energy. Consequently,
the $A'$ decay products will be highly boosted, necessitating a detector with very
forward acceptance that can be placed in close proximity to the target.
Maximizing the acceptance requires placing the detector close to the beam plane,
encroaching on a ``dead zone'' which is occupied by an intense flux of multiple
Coulomb scattered beam particles along with radiative secondaries originating
from the target.  In order to avoid additional background from beam gas
interactions, the detector needs to operate in vacuum. Finally, minimizing the
material budget of the active area of the detector is essential to reducing the
multiple scattering that dominates both the mass and vertex resolutions that
determine the experimental sensitivity.

These design principles led to the conception of the HPS detector.  
Specifically, HPS utilizes a compact, large acceptance forward spectrometer 
consisting of a silicon vertex tracker (SVT) along with a lead tungstate
electromagnetic calorimeter (Ecal).  The SVT is installed inside a vacuum
chamber immediately downstream of a thin ($0.125\%X_{0}$) tungsten target.
The vacuum chamber resides within an analyzing magnet providing a .24 Tesla field 
perpendicular to
the beam plane,
allowing for the precise measurement of track momenta.
%and resides within an analyzing magnet
The Ecal, placed downstream of the tracker, provides the primary 
trigger for the experiment and is also used for electron identification. Together, 
both subsystems provide the complete kinematic information required to 
reconstruct heavy photons.  An overview of the HPS Detector is shown in Figure
\ref{fig:hps_detector}.
\begin{figure}[th]
    \centering
    \includegraphics[width=\textwidth]{images/hps_detector.png}
    \caption{Schematic view of the Heavy Photon Search Detector used during the
             2015 engineering run.}
    \label{fig:hps_detector}
\end{figure}

The HPS detector was installed and commissioned within the Hall B alcove at the
Thomas Jefferson National Accelerator Facility (JLab) in Newport News, VA early
in the spring of 2015. Shortly after, an engineering run took place utilizing
the Continuous Electron Beam Accelerator Facility (CEBAF) operating at an 
energy of 1.056 GeV and current of 50 nA.

The chapter that follows will detail various elements of the experiment.
It will begin with a discussion of CEBAF and continue with descriptions
of several beamline elements, SVT, Ecal and data acquisition system (DAQ).

\section{CEBAF}

CEBAF's ability to provide a nearly continuous, clean and intense electron
beam makes it ideal to search for heavy photons with weak couplings. Recently,
CEBAF underwent an upgrade that increased its maximum operating energy to 12
GeV and introduced a new experimental hall, Hall D, that will house the
GlueX detector \cite{Dudek:2012vr}.  The upgraded facility is now capable of 
delivering 11 GeV electron beams to the three existing experimental halls
(Hall A, B, C) and can use the 12 GeV electron beam to generate and deliver a 9
GeV photon beam to Hall D.  The maximum current that it can deliver to halls
A and C is 85 $\mu A$ while Halls B and D can receive no more than 5 $\mu A$.

As shown in Figure \ref{fig:cebaf}, achieving 12 GeV operation required several
improvements to the accelerator \cite{Burkert:2012rh}. Central to the upgrade 
was the addition of 5 
cryomodules to each of the linacs.  Coupled with upgrades to the accelerator
magnets and power supplies, the additional cryomodules allowed each linac to
accelerate electrons at a rate of 2.2 GeV per pass up to a maximum of 5 passes.
\begin{figure}[h]
    \centering
    \includegraphics[width=0.9\textwidth]{images/cebaf.jpg}
    \caption{A diagram of the Thomas Jefferson National Accelerator Facility
             Continuous Electron Beam Accelerator Facility showing the 
             components that were upgraded as part of the 12 GeV Upgrade 
             program.}
    \label{fig:cebaf}
\end{figure}
Enabling four-hall operation also required the addition of a new 750 MHz RF 
separator, a new laser to the electron source and a 10th arc which provides
the additional pass of acceleration that allows 
delivery of the maximum beam energy to Hall D.


\subsection{Electron Production and Injection}

The electrons injected into the accelerator were produced by photoemission from
a strained GaAs superlattice photocathode \cite{Maruyama:2004hx}.  Each of the 
four experimental halls has a dedicated gain-switched fiber coupled laser of
wavelength 1560 nm.  The lasers are frequency doubled in order to produce light
of wavelength of 780 nm, matching the band gap of the superlattice cathode. The
lasers are phased shifted and are each pulsed for $\approx$ 40 ps at the 
frequency of 499 MHz.  Since the operational frequency of the accelerator 
cryomodules is 1497 MHz, four hall operation requires subharmonics of 499 MHz to
be chosen.  This is achieved by ``cutting away'' pulses using an optical 
modulator \cite{Kazimi:2013yua}.

The photoemission electrons are released into an extremely high vacuum
environment at a pressure of $10^{-11}$ to $10^{-12}$ Torr.  The free electrons
are then delivered into the injector by a 100 keV electron gun.  The injector
itself then accelerates the electron bunches to an energy of 50 MeV 
%by 2 1/4 cryomodules 
before they are delivered into the accelerator.

\subsection{Electron Acceleration}

The CEBAF accelerator is composed of two linacs arranged in a racetrack
configuration as shown in Figure\ref{fig:cebaf}. Each of the linacs consist
of 25 cryomodules, 5 of which were added as part of the upgrade.  The original
(new) cryomodules consist of 8 5-cell (7-cell) superconducting radio frequency
(RF) cavities made of ultra-pure Niobium (see Figure \ref{fig:cebaf_cavity}).  
The original cryomodules
\begin{figure}[h]
    \centering
    \includegraphics[width=0.7\textwidth]{images/cebaf_cavity.jpg}
    \caption{A 5-cell ultra-pure Niobium superconducting radio frequency cavity
             used to accelerate electrons at CEBAF.}
    \label{fig:cebaf_cavity}
\end{figure}
are capable of accelerating an electron upwards of 25 MeV while the newly installed
cryomodules can achieve an acceleration of 100 MeV.  This leads to an acceleration
of 1.1 GeV per linac and 2.2 GeV per pass. The 
number of passes depends on the energy requirements of the experiment taking place.
However, for electrons delivered to Halls A, B and C, the maximum number of passes
is 5, while for Hall D, it's 5.5.

Electron bunches circulating the accelerator can be delivered to a Halls A, B
and C by an RF separator operating at a frequency of 499 MHz.  Delivery to Hall
D uses an RF separator of 750 MHz.

\subsection{Single Pass Operation For HPS}

During the Spring of 2015, HPS was prepared to run at a beam energy of 2.2 GeV,
in conjuction with the commissioning of the 750 MHz RF separator.
Unfortunately, an incident occurred which resulted in the loss of the new CHL
required to operate the accelerator as a 12 GeV machine.
The loss caused the accelerator to 
fallback to 6 GeV operation using a single CHL.  As a result, HPS was given
the unique opportunity to run with a beam energy of 1.056 GeV allowing the
experiment to have sensitivity to the g-2 favored region of the mass-coupling
phase space.  

\begin{sidewaysfigure}
    \centering
    \includegraphics[width=\textwidth]{images/beamline.png}
    \caption{Configuration of the beam line during the HPS engineering run.}
    \label{fig:beamline}
\end{sidewaysfigure}

\section{Beamline}

\subsection{Layout}

The HPS experiment is installed within the Hall B alcove upstream of the CEBAF
Large Acceptance Spectrometer 12 detector.  The experiment utilizes a 
three-magnet chicane system as shown in Figure \ref{fig:beamline}. The distance
between the center of the magnets is 218.1 cm. The second 
dipole of the setup, the Hall B Pair Spectrometer (PS), serves as the analyzing 
magnet of the experiment.  It is a 18D36 magnet with a pole length of 91.44 cm
and gap size of $45.72 \times 15.24 \text{ cm}^2$.  It provides a 0.24 Tesla
field perpendicular to the beam plane. A vacuum chamber, housing the SVT, resides 
within the gap of the analyzing magnet. The first and last dipoles of the 
chicane are ``Frascati'' H-dipole magnets.
They are used to steer the deflected beam back onto the beam line and into the
beam dump downstream.  
They have a pole length of 50 cm and were operated at 0.6 Tesla.

A 0.125\%X$_{0}$ ($\sim 4\mu$m thick) tungsten target is installed at the
upstream edge of the 
PS magnet, 10 cm from the first layer of the SVT.
The target is connected a rigid support rod that is attached on the upstream 
side to a linear shift.  This allows the precise movement of the target
in an out the of the beam plane using a stepper motor.  Far upstream of the target, a 10 mm
thick tungsten protection collimator with a $4\times 10 \text{ mm}^2$ hole
was installed. The protection collimator was used to protect the SVT
silicon planes from being hit directly by mistered beam.

A vacuum box is mounted on the upstream end of the PS vacuum 
chamber and is used to provide vacuum penetration for linear motion system, 
cooling lines, power and signal cables (see Figure \ref{fig:svt_layout_render}.
On the downstream end of the PS vacuum 
chamber, another vacuum chamber is installed and is encroached by the upper
and lower modules of the SVT.  

%\subsection{Beam Quality}

\section{Silicon Vertex Tracker}

\subsection{Layout}

The HPS SVT is comprised of two halves of six measurement layers encroaching the
beam plane as shown of Figure \ref{fig:svt_layout_render}. Each layer consist of a pair of
closely-spaced silicon strip detector planes with one of the planes oriented orthogonal to
the beam plane and the other at small angle stereo 
(see Table \ref{tab:svt_layout}).
\begin{figure}[h!t]
    \centering
    \includegraphics[width=0.9\textwidth]{images/svt_layout_render.png}
    \caption{A rendered view of the Silicon Vertex Tracker inside the pair
             spectrometer vacuum chamber.}
    \label{fig:svt_layout_render}
\end{figure}
This allows
for the measurement of both the vertical and horizontal coordinate of a hit, in turn, 
enabling full 3D hit reconstruction.  

The first three layers consist of a single sensor of coverage above and below
the beam plane and use a stereo angle of 100 mrad. In order to better match the
% needs a sentence explaining why 100 mrad stereo was used.  According to the
% proposal, it balances acceptance against vertexing resolution.
acceptance of the Ecal, the coverage of the last three layers is two sensors 
wide and use a stereo angle of 50 mrad.  The choice of a 50 mrad angle for the
last three layers instead of 100 mrad was meant to break the degeneracy that
results in fake tracks
due to ghost hits in layers with the same stereo angle.  It must be noted that
only five layers are needed to match the full acceptance of the Ecal, however,
an improvement in the momentum resolution was observed with the addition of 
another layer.  In total, the SVT makes use of 36 sensors, which amounts to 
23,004 channels.

Since heavy photons are produced very forward, and the opening angle of their 
decay products goes as $\sim m_{A}/E_{0}$, sensitivity to low mass heavy photons
requires
the tracker layers to be as close to the beam plane as possible.  When deciding
the distance of the first layer to the beam, several effects needed to be taken
into consideration. These include the extent of the beam halo, the amount of 
radiation damage that is expected to be incurred from the Coulomb scattering
of the primary beam as well as radiative secondaries, the ability to resolve 
hits with pileup present and being capable of doing pattern recognition in a high 
occupancy environment.  With all of this in mind, it was determined that the 
closest tolerable angular proximity was 15 mrad, putting the edge of layer 1 
at 0.5 mm from the beam center. In simulation, this corresponds to 1\% occupancy
of strips closest to the beam plane of layer 1.

% I should talk about the positions as surveyed.
%The SVT layout is summarized in Table \ref{tab:svt_layout}. \textbf{Talk about
%the survey}.

%%%%%%%%%%%%%%%%%%%%%%%%%%%
%%% Table of SVT layout %%%
%%%%%%%%%%%%%%%%%%%%%%%%%%%
% I should use the survey positions here.
%\begin{table}[t]
\begin{sidewaystable}
    \centering
    \begin{tabular}{lcccccc}  
        \toprule
        \textbf{Layer} & \textbf{1} & \textbf{2} & \textbf{3} & \textbf{4} & \textbf{5} & \textbf{6} \\
        \midrule
        \midrule
        $z$ position from target (cm)    & 10 & 20 & 30 & 50 & 70 & 90 \\
        Stereo angle (mrad) & 100 & 100 & 100 & 50 & 50 & 50 \\
        Bend plane resolution ($\mu$m) & $\approx$6 & $\approx$6 & $\approx$6 & $\approx$6 & $\approx$6 & $\approx$6 \\
        Non-bend plane resolution ($\mu$m) & $\approx60$ & $\approx60$ & $\approx60$ & $\approx120$ & $\approx120$ & $\approx120$ \\
        Nominal dead zone in $y$ (mm) & $\pm$ 1.5 & $\pm$ 3.0 & $\pm$ 4.5 & $\pm$ 7.5 & $\pm$ 10.5 & $\pm$ 13.5 \\ 
        Material budget (\%$X_0$) & .7 & .7 & .7 & .7 & .7 & .7 \\
        \bottomrule
    \end{tabular}
    \caption{The layout of the HPS SVT.}
    \label{tab:svt_layout}
\end{sidewaystable}
%%%%%%%%%%%%%%%%%%%%%%%%%%%

\subsection{Sensors}

At the energies at which HPS operates, the uncertainty in both the mass and
vertex resolutions are dominated by multiple Coulomb scattering in the first 
few layers.  This made it important to choose a sensor technology that would 
minimize the material budget of the SVT modules, especially since the material
budget of the sensors dominates the total material budget of the SVT modules.
Furthermore, the need to place the SVT in close proximity
to the beam plane made it necessary to choose sensors which are highly tolerant
to radiation.  With these 
considerations in mind, a readily available batch of silicon microstrip sensors,
initially manufactured for the D0 Run IIb upgrade, were found to satisfy all 
necessary requirements \cite{D0Collab:2003}.

The sensors were manufactured by Hamamatsu Photonics Corporation on 
$\langle 100 \rangle$ crystal rotation silicon and are $p^{+}$ on $n$-bulk, 
single sided, AC-coupled and polysilicon-biased. The cut dimensions of the 
sensors are $100 \times 40.34$ mm$^{2}$ with an active area of 
$98.33 \times 38.34$ mm$^{2}$. They are $320 \pm 20 \mu$m thick and have a sense
(readout) pitch of 30 (60) $\mu$m. The sensor specifications are summarized on 
Table \ref{tab:sensor_specs}.
\begin{table}[t]
    \centering
    \begin{tabular}{lr}
        \toprule
        Cut dimensions (L$\times$W)     & 100 mm x 40.34 mm \\
        Active area (L$\times$W)        & 98.33 mm x 38.34 mm \\
        Readout (Sense) pitch           & 60 (30) $\mu$m \\
        \# Readout (Sense) strips       & 639 (1277) \\
        Breakdown voltage               & $>1000$ V \\
        Depletion voltage               & $> 130$ V \\
        Bias Resistor Value             & $0.8 \pm 0.3$ M$\Omega$ \\
        AC Coupling Capacitance         & $>12$ pF/cm \\
        Total Interstrip Capacitance    & $< 1.2$ pF/cm \\
        Defective Channels              & $<1$ \% \\
        \bottomrule
    \end{tabular}
    \caption{Specifications of the sensors used for the HPS SVT.}
    \label{tab:sensor_specs}
\end{table}

Over the lifetime of the HPS detector, the sensor strips closest to the beam 
plane are expected to see $>10^{15}$ electrons per cm$^2$.  The radiation
damage the sensors are expected to incur due to the large electron flux
will lead to an increase in both the leakage current and the voltage required to 
fully deplete the sensor.  It was thus beneficial to choose a sensor technology
that can be operated at high bias voltage in order for them to remain fully
depleted even after irradiation. In fact, previous studies have shown that 
sensors that may be operated to 1000 V can tolerate a dose of 
$1.5 \times 10^{14}$ 1 MeV neq/cm$^2$ \cite{Fretwurst:2002vb}.  Since the damage
incurred by electrons with energies less than 10 GeV is a factor $\sim$ 30 less
than 1 MeV neutrons \cite{Rashevskaya:2002nd},
then ensuring all sensors can be biased to 1000 V will ensure that the sensors
will be able to withstand the expected flux of electrons over the lifetime of
HPS. 


Before being considered for use for the SVT modules, all sensors were electrically 
characterized.  Specifically, the leakage current was measured as a function
of bias voltage up to a maximum bias of 1000 V.  During these test, leakage 
currents of less than 500 nA were observed.  The measured IV curves for a subset
of sensors can be seen on Figure \ref{fig:sensor_iv_curves}.  Only sensors whose
leakage current did not uncontrollably increase (i.e. break down) before reaching
a bias of 1000 V  were considered for use in HPS.
\begin{figure}[t!]
    \centering
    \includegraphics[width=\textwidth]{images/sensor_iv_curves.png}
    \caption{Measured IV curves before irradiation for a subset of sensors used by HPS.}
    \label{fig:sensor_iv_curves}
\end{figure}

\subsection{Readout} \label{subsec:readout}

The sensors are continuously read out using the APV25 readout chip developed for
the Compact Muon Solenoid detector at the Large Hadron Collider 
\cite{Raymond:2000ey}. The APV25 has 128 channels, with each channel consisting
of a charge-sensitive pre-amplifier coupled to CR-RC shaping amplifier and a 
192-cell-deep analog pipeline.  A schematic of a single channel is shown in 
Figure \ref{fig:apv25_schem}.
\begin{figure}
    \centering
    \includegraphics[width=\textwidth]{images/apv25_channel_schematic.png}
    \caption{A schematic of a single channel of the APV25 readout chip.}
    \label{fig:apv25_schem}
\end{figure}

When a particle traverses a sensor, it generates a charge signal which is 
processed by the APV25 amplifier chain.  As shown in Figure 
\ref{fig:apv25_pipeline}, the shaper
\begin{figure}
    \centering
    \includegraphics[width=\textwidth]{images/apv25_pipeline.png}
    \caption{A schematic demonstrating the sampling of the shaper signal and
             the management of read/write pointers.}
    \label{fig:apv25_pipeline}
\end{figure}
output is continuously sampled at 41.6 MHz into the analog pipeline. The position
along the pipeline into which the shaper output is stored is determined by 
a write pointer which continuously cycles through the pipeline.  Similarly, a read 
pointer determines the position that will be marked for read out when a trigger
signal is received.  Since the trigger decision cannot happen instantaneously,
the distance between the read and write pointers or latency is programmable.
Given that only 160 pipeline cells out of the 192 are used
to buffer samples, the delay between a signal and the arrival of the trigger can
be as long as 3.8 $\mu s$. The remaining 32 cells along the pipeline are used 
to buffer the addresses of samples that are waiting to be read out.

The samples  are read out by the Analog Pulse Shape Processor (APSP) which
can operate in two modes: multi-peak and deconvolution mode.  In deconvolution
mode, three consecutive pipeline cells are read out and combined into a 
weighted sum before being output. In multi-peak mode, three consecutive samples are
read out and output without any additional operations, allowing for the 
reconstruction of the shaper output.  The output of 
the APSP is then sent to a 128:1 multiplexer which makes the raw data 
frames. 
%An example of the output data frame is shown in Figure \ref{}.

During the engineering run, the APV25s were operated using the nominal settings
listed in Table \ref{tab:apv_specs}.  The nominal shaping time is set
to 50 ns.  The high occupancies expected during the engineering
run meant that overlapping of hits or ``pile-up'' were a concern.  In order to 
mitigate this problem, the APV25s were operated in multi-peak mode allowing 
the reconstruction of the shaper output.  Furthermore, 
with each Ecal trigger, the APV25s were sent two consecutive trigger signals 
allowing the readout of six consecutive samples instead of three.  The trigger
latency was then adjusted such that two samples before the signal were read out, 
allowing the shape of the pileup pulse to be captured by the fit.  This was 
used to remove any effects of pileup from the signal pulse.  Approximately 5\%
of hits in layer 1 were observed to be affected by pileup.
\begin{table}[ht]
    \centering
    \begin{tabular}{llr}
        \toprule
        \textbf{Name} & \textbf{Description} & \textbf{Value} \\
        \midrule
        \midrule
        IPRE   & Preamp input FET current       & 98 (460 $\mu$A)\\
        IPCASC & Preamp cascode current         & 52 (60 $\mu$A) \\
        IPSF   & Preamp source follower current & 34 (50 $\mu$A) \\
        ISHA   & Shaper input FET current bias  & 34 (50 $\mu$A) \\
        ISSF   & Shaper source follower current & 34 (50 $\mu$A) \\
        IPSP   & APSP current                   & 55 (80 $\mu$A) \\
        IMUXIN & Multiplexer input current      & 34 (50 $\mu$A) \\
        VFP    & Preamp feedback voltage        & 30  \\
        VFS    & Shaper feedback voltage        & 60  \\
        VPSP   & APSP Voltage level             & 40  \\
        LATENCY & Trigger latency               & 147 \\
        MUXGAIN & Multiplexer gain              & 100 $\mu$A/MIP \\
        \bottomrule
    \end{tabular}
    \caption{APV25 specs used during the engineering run.}
    \label{tab:apv_specs}
\end{table}
%\ref{fig:apv_shape}.
%\begin{figure}
%    \centering
%    \includegraphics[width=0.9\textwidth]{images/sideB_response_ch535.png}
%    \caption{The signal shape observed during the engineering run.}
%    \label{fig:apv_shape}
%\end{figure}

\subsection{SVT Modules}

Two different sensor module designs were used for the layers of the SVT.
A module was built by sandwiching a pair of half-modules around an aluminum 
cooling block located on the hybrid side. For layers 1-3, the half-modules are 
fixed on the hybrid side while a spring and lever mechanism tensions them on
the opposite end.  For layers 4-6, the tensioning mechanism is integrated into
into one of the cooling blocks.

The first three layers of the SVT reused the half-module design from the HPS 
test run \cite{Battaglieri:2014hga}. They consist of a single sensor
and FR4 hybrid electronic board glued onto a polyimide-laminated carbon fiber
composite backing. The hybrid contains filtering for the high voltage bias, temperature
sensors and the APV25 readout chips used to read out the sensor. The APV25 chips
are mounted 
onto the hybrid electronic board and the channel pads are wiredbonded directly
to the sensor. Each sensor requires 5 APV25 chips in order to read out all channels.
In order to further minimize the material budget of the half-modules, 
a window 
was machined into the carbon fiber, leaving the middle of the sensor exposed.

In order to better match the acceptance of the Ecal, 
the half-modules used in layers 4-6 consist of two sensors glued end-to-end onto
the polyimide-laminated carbon fiber backing with hybrids on either side of them.
Due to space constraints, the hybrids used by layers 4-6 have a smaller footprint
compared to the hybrids used for layers 1-3.

%Each sensor requires 5 APV25 chips in order to readout all channels.  The 5 
%APV25 chips are mounted on a FR4 electronic readout board, or hybrid, containing
%filtering for the high voltage bias and a temperature sensor.  Since the pitch
%of the APV25 and the sensors are similar, the chips were wirebonded directly to
%the sensors without the need of a pitch adapter.

\begin{figure}
    \centering
    \includegraphics[width=0.9\textwidth]{images/l13_half_module.jpg}
    \caption{A layer 1-3 half-module used by the SVT. }
    \label{fig:l13_hm}
\end{figure}
\begin{figure}
    \centering
    \includegraphics[width=0.9\textwidth]{images/l46_half_module.jpg}
    \caption{A layer 4-6 half-module used by the SVT. }
    \label{fig:l46_hm}
\end{figure}


\subsection{Mechanical Support, Cooling and Services}

The SVT modules are mounted directly to 4,  1/4'' aluminum ``u-channels'' with each
of the u-channels supporting 3 modules as shown in Figure \ref{fig:l46_hm}.  
The u-channels are actively cooled by HFE 7000 flowing through 1/4'' copper 
tubing press-fit into pre-machined grooves into the aluminum. Each of the 4 
u-channels are installed inside an aluminum support box using a guide rail system
allowing quick access to the mounted modules 
%(see Figure \ref{})
.  The layer 
4-6 u-channels are fixed inside of the box while the layer 1-3 u-channels 
have a rigid support rod attached on the upstream end which are connected to 
linear shifts. The linear shifts are used to precisely move the layer 1-3 
u-channels vertically in 6 $\mu$m steps using stepper motors.  This, in turn, 
allows the placement of the edge of the layer 1 sensor at $\sim$ 7 mm from the 
beam plane.  Each of the two layer 1-3 u-channels are connected to their own 
linear shifts.  The support box is installed inside the Hall B analyzing magnet 
vacuum enclosure. 

The support box is also used to house an aluminum plate onto which data 
acquisition boards are mounted.  The aluminum plates slides into the support box
via a machined grove.  An embedded loop is used to circulate water through the
plate, providing cooling of the boards.


\section{Electromagnetic Calorimeter}

The HPS Ecal is used to provide the primary trigger for the experiment as well as to
identify electrons.  It consist of two halves of lead-tungstate 
(PbWO$_4$) crystals with each half mounted on an aluminum frame $\sim 137$ cm 
from the upstream edge of the analyzing magnet.  Each half is composed of five
layers of crystals with the four outermost layers consisting of 46 crystals and 
\begin{figure}
    \centering
    \includegraphics[width=0.8\textwidth]{images/ecal_layout.png}
    \caption{A rendering showing the arrangement of the Ecal crystals.  The Ecal
             is split into upper and lower modules in order to accommodate the 
             ``dead zone.''  The crystals removed from the first layer allow
             a larger opening for the outgoing electron and photon beams.}
    \label{fig:ecal_layout}
\end{figure}
the layer closest to the beam plane consisting of 37. The removal of the 9 
crystals from the inner layer was necessary to allow the outgoing electron and
photon beams to pass through unimpeded.  Each half is enclosed in a temperature
controlled environment held at 17$^{\circ}$C which encroaches on the Ecal 
vacuum chamber.

Each of the crystals is 16 cm long and trapezoidal in shape with a front face
dimension of $1.3 \times 1.3$ cm$^2$ and a back face dimension of $1.6 \times
1.6$ cm$^2$.  In order to maximize the light yield, the crystals were wrapped
in VM2000 non-metallic reflector film. A Hamamatsu S8664-1010 Avalanche 
Photodiode (APD) with a photosensitive area of $10 \times 10$ mm$^2$ was glued
to the back of each crystal and used to read out the signals collected by the
crystals.  
\begin{figure}
    \centering
    \includegraphics[width=0.8\textwidth]{images/ecal_crystal.png}
    \caption{Rendered view of an HPS Ecal module consisting of a 16 cm PbW$_4$
             crystal, Avalanche Photodiode and preamplifier board.}
    \label{fig:ecal_crystal}
\end{figure}

\section{Trigger and Data Acquisition}

\subsection{Ecal Data Acquisition}

The analog signals that are read out from each of the Ecal crystals by the APDs
are sent to a 16-channel JLab FADC250 VXS module (FADC)
(see Figure \ref{fig:ecal_fadc}).  The 221 FADC channels used by each half of 
the Ecal are housed in their own 20 slot VSX crates.
\begin{figure}
    \centering
    \includegraphics[width=0.7\textwidth]{images/ecal_fadc.png}
    \caption{A 16-channel Jefferson Lab FADC250 VXS module.}
    \label{fig:ecal_fadc}
\end{figure}

The APD signals are sampled and digitized by the FADCs at a rate of 250 MHz
into 8 $\mu$s deep pipelines.  If an FADC signal crosses a pre-defined
threshold, the integrated amplitude of a select number of samples before and
after the threshold crossing, as well as the crossing time, are passed to the
Crate Trigger Processor (CTP).

\subsection{Trigger}

The HPS trigger is designed to efficiently select $e^+e^-$ pairs whose energy
depositions, or clusters, in the Ecal are consistent with 
%with either a trident reaction or 
the decay of an $A'$. The trigger logic searches for signals that
are coincident in time and satisfy a specific kinematic selection optimized
to select $A'$ events.

As discussed in the previous section, if a signal from an Ecal crystal is found
to cross some pre-determined threshold, the crossing time and amplitude are 
reported to the CTP. The CTP contains the cluster finding algorithm
which performs the following task: 
\begin{itemize}
    \item The amplitude of hits from every 3x3 array of crystals in the Ecal that
          is within a programmable number of clock cycles is summed. 
    \item If the 3x3 sum exceeds a pre-defined cluster amplitude threshold and
          the sum is greater than any of the neighboring 3x3 windows, then the 
          amplitude (energy), position, time and hit pattern are reported to the Sub-System
          Processor (SSP).
\end{itemize}

The SSP takes the cluster information reported by both halves of the Ecal and
creates all possible pairs of clusters that fall within an 8 ns coincident
window.  Then, in order to further reduce background rates, the following 
selection is applied to the pairs of clusters: 
\begin{itemize}
    \item $E_{min} \le E_{top} + E_{bottom} \le E_{max}$
    \item $| t_{top} - t_{bottom} | \le \Delta t_{max}$
    \item $|E_{top} - E_{bottom} \le \Delta E_{max}$
    \item $E_{low} + R \times  F \le$ Threshold$_{slope}$
    \item $|\tan^{-1}\frac{X_{top}}{Y_{top}} - \tan^{-1}\frac{X_{bottom}}{Y_{bottom}}| \le \theta_{Coplanarity}$
\end{itemize}
Here, $E_{top}$ ($E_{bottom}$), $t_{top}$ ($t_{bottom}$), $x_{top}$ 
($x_{bottom}$) and $y_{top}$ ($y_{bottom}$) are the energy, timestamp and 
position of the cluster in the top (bottom) half of the Ecal and $E_{min}$ 
($E_{max}$) is the minimum (maximum) cluster energy sum. $E_{low}$ is the 
energy of the lowest energy cluster, $R$ is the distance between its 
center and the calorimeter center while $F$ is a constant.
%As shown in Table \ref{tab:triggers}, several of these parameters are programmable.
The values used during the engineering run are 
listed in Table \ref{tab:triggers}. If a pair of clusters
satisfies these criteria, a trigger signal is generated by the Trigger Supervisor
and sent to all subsystems. 

During the engineering run, several triggers were run simultaneously.  The main
trigger used to select $A'$ type events is the Pair-1 trigger.  The Pair-0 
trigger is a much looser version of the Pair-1 trigger and was tuned to select 
electron-electron elastic scattering (M\o ller scattering) events.  The Single-1 trigger was tuned to select electrons
that Coulomb scatter in the target i.e. full energy electrons (FEE) into the 
acceptance of the Ecal.  These events are used to study both the momentum 
resolution of the tracker and the energy resolution of the Ecal.  Finally, 
there was a cosmic trigger and a pulser trigger used to trigger on cosmic ray
muons and randoms respectively.  A summary of all of the settings is given
in Table \ref{tab:triggers}

\begin{table}
    \centering
    \begin{tabular}{lcccc}
        \toprule
        \textbf{Parameter} & \textbf{Single-0} & \textbf{Single-1} & \textbf{Pair-0} & \textbf{Pair-1} \\
        \midrule
        \midrule
        $E_{min}$ (GeV)      & 0.060 & 0.400 & 0.054 & 0.054 \\
        $E_{high}$ (GeV)     & 2.500 & 1.100 & 1.100 & 0.630 \\
        $N_{threshold}$      & 3     & 3     & 1     & 1     \\
        $E_{sum low}$ (GeV)  &       &       & 0.120 & 0.180 \\
        $E_{sum high}$ (GeV) &       &       & 2.000 & 0.860 \\
        $E_{\text{differenec}}$ (GeV) &       &       & 1.000 & 0.540 \\
        $F (\text{GeV})$ & & & & 0.0055 \\
        $\theta_{\text{coplanirity}}$ & & & & 30$^\circ$ \\
        $t_{\text{coplanirity}} (\text{ns})$ & & & 16 & 12 \\
        Prescale & $2^{13}$ & $2^{11}$ & $2^{10}$ & $2^{0}$ \\
        Rate (50 nA) & 0.4 Hz & 1.3 kHz & 0.7 kHz & 16.6 kHz \\
        \bottomrule
    \end{tabular}
    \caption{The trigger setting for all trigger types used during the 
             engineering run. The pair-1 trigger was the main trigger used 
             by the experiment.}
    \label{tab:triggers}
\end{table}

\subsection{SVT Data Acquisition}
% Need to discuss the hybrid design for both layers 1-3 and 4-6.
% 1) Need to mention why different wiring was used for the different parts of 
% the SVT.

After a trigger is received, the differential current signals 
from each of the APV25s are transferred to a total of 10 Front End Boards (FEB)
to undergo digitization and further processing.  The FEBs contain all of the
necessary electronics to digitize the signal and to distribute power to
the hybrids.
%(Add figure?)
The signals from layers 1-3 are transferred to the FEBs via 
Teflon-coated twisted pair wires while those emerging from
layers 4-6 use twisted pair magnet wire.  The use of twisted pairs reduces
crosstalk between the lines as well as electromagnetic interference.

At the FEB's, the differential current signals are first converted to a voltage
by a pre-amplifier circuit to match the dynamic range 
of the AD9252 14-bit analog to digital converter (ADC). The ADC samples the 
signal at 41.667 MHz and digitizes it to a value between 0 and 16384.  The 
digitized signals are then transferred to Xilinx Artix-7 field programmable
gate arrays (FPGA) which sends the signals upstream to multi-gigabit transceivers.
In addition to transmitting data upstream, the FPGAs are also responsible for 
controlling and monitoring the hybrid power and distributing triggers and 
clock to the hybrids.

Data from the FEBs are transferred through 
mini SAS cables to electronic boards potted through slots on a 8 inch vacuum 
flange located upstream on the vacuum box. The flange boards contain the necessary
electronics to convert the digitized signal to optical.
The optical signal is then transferred over ~30 m fibers to an ATCA crate. The
data from all 10 FEBs is distributed between two
ATCA blades, called Cluster on board (COB), housed inside the ATCA crate.
Each COB contains
8 processing nodes known as Reconfigurable Cluster Elements (RCE). The 
processing nodes use Xilinx Zynq-7000 series FPGAs to apply data reduction
algorithms to the incoming signals and build event frames.  



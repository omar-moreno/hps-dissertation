
\chapter{The HPS Apparatus}

\section{Introduction}

\section*{Silicon Vertex Tracker}

\subsection*{Design Cosiderations}

Since the electroproduced $A'$ will carry most of the beam energy, the opening
angle of low mass dark photons will be quite small.  Consequently, the $A'$ 
decay products will be highly boosted, requiring a detector with very forward
acceptance that can be placed in close proximity to the beam plane, which is
occupied by the intense flux of multiple Coulomb scattered beam particles and
radiative secondaries originating from the target.  This establishes a 
``dead zone'' that the detector  is unable to encroach in  order  to avoid  
extensive  radiation damage.  Furthermore, the detector must also be operated 
in vacuum in order to avoid additional background from beam gas interactions.
Finally, minimizing the material budget of the active area of the detector is
essential to reducing multiple scattering that dominates both the mass and 
vertex resolutions that determine the experimental sensitivity.

\subsection*{Layout}

The SVT is comprised of six measurement layers, each consisting of a pair of
closely-spaced silicon planes as shown in Fig. ().  A stereo angle is 
introduced between the two planes within each layer allowing for the 
measurement of both the vertical and bend coordinate of a hit, in turn, 
enabling full 3D hit reconstruction.  The first three layers consist use
a stereo angle of 100 mrad. In order to better match the acceptance of the 
Ecal, the coverage of the last three layers is two sensors wide and use a 
stereo angle of 50 mrad.  In total, the SVT will make use of 36 sensors, which
amounts to 23,004 channels.  The SVT layout is summarized in Table ().

%%%%%%%%%%%%%%%%%%%%%%%%%%%
%%% Table of SVT layout %%%
%%%%%%%%%%%%%%%%%%%%%%%%%%%
\begin{table}[t]
\begin{center}
\begin{tabular}{l|cccccc}  
Layer & 1 & 2 & 3 & 4 & 5 & 6 \\ \hline
$z$ position from target [cm]    & 10 & 20 & 30 & 50 & 70 & 90 \\
Stereo angle [mrad] & 100 & 100 & 100 & 50 & 50 & 50 \\
Bend plane resolution [$\mu$m] & $\approx$6 & $\approx$6 & $\approx$6
& $\approx$6 & $\approx$6 & $\approx$6 \\
Non-bend plane resolution [$\mu$m] & $\approx60$ & $\approx60$ & $\approx60$
& $\approx120$ & $\approx120$ & $\approx120$ \\
\hline
\end{tabular}
\caption{The proposed layout of the SVT.}
\label{tab:svt_layout}
\end{center}
\end{table}
%%%%%%%%%%%%%%%%%%%%%%%%%%%


The SVT is split into upper and lower tracking volumes in order to avoid
the 15 mrad ``dead zone'', putting the active area of the sensors at 1.5 mm
from the beam plane.  The layers are mounted on upper and lower support 
structures that are hinged on the downstream end of the SVT.  This allows
adjustment of the vertical position of the sensors remotely by a motion
control system, in response to experimental conditions.

\subsection{Components}

\section{Electromagnetic Calorimeter}

The  Ecal  was  used  as  the  primary  trigger  for  the  experiment  as  well  as  for
electron identication. 

\subsection{Layout}

\section{Trigger and Data Acquisition}


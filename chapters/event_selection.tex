%
%
%
%
\chapter{Event Selection}

Searching for a heavy photon resonance requires both the accurate 
reconstruction of the QED trident invariant mass spectrum and the efficient 
rejection of Bethe-Heitler events.  With this in mind, a scheme 
was developed to select events with $e^+e^-$ pairs that have well 
reconstructed clusters and tracks associated with them.  Additional kinematic 
requirements were applied to reject Bethe-Heitler events.  The  $e^+e^-$ events
which satisfied all criteria were used to produce the final invariant mass 
spectrum employed to search for a heavy photon resonance.  The following chapter
will discuss the data and selection used to arrive at the final event sample.   

\section{Data}

The data used for this analysis consist of the unblinded portion of the 2015 
HPS engineering run.  This amounts to approximately $\sim$10\% of the total 
data collected during the run.  The list of runs used in this analysis along
with the total number of events and luminosity for 10\% of each run 
are shown on Table \ref{tab:data}.
%%%%%%%%%%%%%%%%%%%%%
%   Table of Data   %
%%%%%%%%%%%%%%%%%%%%%
\begin{table}[t]
    \begin{center}
        \begin{tabular}{ccc}
            \hline
            \textbf{Run Number} & \textbf{Total Events (M)} & \textbf{Luminosity (nb$^{-1}$)} \\
            \hline
            5723 & 10.68765  & 3.927681901  \\
            5724 & 11.397637 & 4.229082832    \\
            5725 & 8.612363  & 3.193363425   \\
            5739 & 8.0932    & 2.956111602   \\
            5741 & 10.97741  & 3.987840806   \\
            5742 & 11.174619 & 4.224543485   \\
            5743 & 6.040229  & 2.244021498 \\
            5766 & 9.927829  & 3.337962213 \\
            5769 & 11.179994 & 4.059284573 \\
            5771 & 11.642768 & 4.254619178 \\
            5772 & 11.865312 & 4.403638304 \\
            5773 & 13.429284 & 5.014898695 \\
            5775 & 5.580172  & 1.988647799 \\
            5776 & 8.121968  & 2.997741852 \\
            5782 & 12.396243 & 4.552604162 \\
            5783 & 12.226626 & 4.548117122 \\
            5791 & 10.58427  & 3.716081116 \\
            5795 & 8.789774  & 3.17431154 \\
            5796 & 12.055927 & 4.323879491 \\
            5797 & 10.047503 & 3.584285692  
        \end{tabular}
    \end{center}
    \label{tab:data}
\end{table}

Before being considered for the final event selection discussed later in the chapter, 
all events were required to satisfy the ``pair1'' trigger criteria 
discussed in section [].  Furthermore, only events where the bias of the SVT was
on, the SVT was positioned at 0.5 mm from the beam and were free of data 
acquisition errors were considered.

\section{Event Selection}

\subsection{Cluster Pair Selection}

The acceptance of the HPS detector is optimized such that the $e^+e^-$ pairs
produced in either a QED trident reaction or from the decay of an 
$A'$ are observed through their energy depositions in the Ecal.  The energy
depositions or clusters are expected to be in opposite Ecal volumes i.e. 
top/bottom and coincident in time within a coincidence window of a few ns.  This
can be seen from Fig. \ref{fig:cluster_times_2d}, which plots the time relative to 
the trigger time (cluster time) of one cluster composing a pair versus the 
other.  From the figure, it can be seen that most cluster pairs of interest are 
%%%%%%%%%%%%%%
%   Figure   %
%%%%%%%%%%%%%%
\begin{figure}[t]
    \centering
    \includegraphics[width=.9\textwidth]{images/20160428_pass4_cluster_time_v_cluster_time.png}
    \caption{The Ecal cluster time of one cluster versus it's corresponding pair for all
             pairs in an event.}
    \label{fig:cluster_times_2d}
\end{figure}  
%%%%%%%%%%%%%%
%   Figure   %
%%%%%%%%%%%%%%
coincidence to within a few ns. Furthermore, as can be seen from Fig. 
\ref{fig:cluster_times}, 
%%%%%%%%%%%%%%
%   Figure   %
%%%%%%%%%%%%%%
\begin{figure}[t]
    \centering
    \includegraphics[width=.9\textwidth]{images/20160428_ecal_cluster_time.png}
    \caption{The time relative to the trigger time of all Ecal clusters.}
    \label{fig:cluster_times}
\end{figure}  
%%%%%%%%%%%%%%
%   Figure   %
%%%%%%%%%%%%%%
all clusters also fall within a cluster time window. 

In order to select true coincidences, the clusters forming a pair were 
required to have a cluster time between 42 ns and 47.5 ns.  The efficiency
of finding a track associated with a cluster was found to dramatically drop 
for clusters found outside of this window.  The selection is
shown graphically in red on Fig. \ref{fig:cluster_times}.  Clusters satisfying
the cluster time criteria are then formed into pairs.  The coincidence timing
between a pair is then required to fall within a 3.2 ns window around the 
coincidence peak at 0.003 ns (Fig. \ref{fig:coin_time}).  If an event contains
multiple ``good'' cluster pairs, the pair with the smallest coincident time is 
chosen. 
%%%%%%%%%%%%%%
%   Figure   %
%%%%%%%%%%%%%%
\begin{figure}[t]
    \centering
    \includegraphics[width=.9\textwidth]{images/20160428_coincidence_time.png}
    \caption{Coincidence time between two clusters in an event.}
    \label{fig:coin_time}
\end{figure}  
%%%%%%%%%%%%%%
%   Figure   %
%%%%%%%%%%%%%%
 
\subsection{Track-Cluster Matching}
   
In order to further supress accidental cluster pairs, the 
selected ``good'' cluster pairs are required to have tracks associated 
with them. The trajectories of all tracks in an event are propogated downstream
to the face of the Ecal using the full magnetic field map.  A track and a cluster
are considered a match if the difference between their positions in both $x$ 
and $y$ satisfy the criteria listed on Table \ref{tab:track_cluster_cuts}. The 
cuts were chosen such that they create a 3$\sigma$ window around the peak of 
the cluster-track position difference distributions 
(See Figs. \ref{fig:track_cluster_delta_x} and \ref{fig:track_cluster_delta_y}).
\begin{figure}[h]
    \begin{subfigure}{.5\textwidth}
        \centering
        \includegraphics[width=0.85\textwidth]{images/20160502_pass4_cluster_track_delta_x_top.png}
    \end{subfigure}
    \begin{subfigure}{.5\textwidth}
        \centering
        \includegraphics[width=0.85\textwidth]{images/20160502_pass4_cluster_track_delta_x.png}
    \end{subfigure}
    \caption{Histogram}
    \label{fig:track_cluster_delta_x}
\end{figure}
\begin{figure}[h]
    \begin{subfigure}{.5\textwidth}
        \centering
        \includegraphics[width=0.85\textwidth]{images/20160502_pass4_cluster_track_delta_y_top.png}
    \end{subfigure}
    \begin{subfigure}{.5\textwidth}
        \centering
        \includegraphics[width=0.85\textwidth]{images/20160502_pass4_cluster_track_delta_y_bottom.png}
    \end{subfigure}
    \caption{Histogram}
    \label{fig:track_cluster_delta_y}
\end{figure}
If multiple tracks satisfy this criteria, the track 
\begin{table}[h]
    \begin{center}
        \begin{tabular}{r|cc}
            \hline
                    & Cluster x - track x at Ecal (mm) &  Cluster y - track y at Ecal (mm)  \\
            \hline
            Top     & $\ge -6.10$ and $\le 12.93$ & $\ge -6.08$ and $\le 11.49$ \\ 
            Bottom  & $\ge -8.02$ and $\le 10.84$ & $\ge -8.31$ and $\le 7.40$  \\ 
            \hline
        \end{tabular}
    \end{center}
    \caption{Table with cuts.}
    \label{tab:track_cluster_cuts}
\end{table}
closest to the cluster is chosen.  Currently, there are no requirements on the
ratio of cluster energy to the track momentum, $E/p$.
   
\subsection{Final Trident Sample}

The event selection criteria that have been applied thus far only select 
events that have QED trident or $A'$ like signatures i.e. two coincident clusters that
satisfy trigger cuts have an $e^+e^-$ track associated with them.  However, 
there remains several accidentals that need to be removed from the final 
event sample.  This is best accomplished by subjecting the tracks associated 
with the clusters to additional criteria.  The criteria are as follows: 
\begin{itemize}
    \item For simplicity, events that have multiple positron tracks are not 
          considered.
    \item In order to cut down on the number of misconstructed tracks that may
          have been mismatched to a cluster, both tracks are subjected to 
          $\chi^2$ probability cut of 95\%.
    \item Some of the electrons in the cluster pair may actually be a multiple
          Coulomb scattered electron instead of one assicated with an $e^+e^-$ 
          pair.  To remove these events from the final event sample, the 
          momentum of electron tracks are required to be less than 0.85 GeV.
    \item The $e^+e^-$ tracks associated with the clusters are vertexed with 
          their position along the beamline, $v_z$, constrained to the target. 
          For true $e^+e^-$ pairs, the vertex position should be well constrained
          to an ellipse.  With this in mind, 
          the $chi^2$ of the vertex fit is required to be less than 10, while
          the positions along x and y ($v_x$ and $v_y$) must lie within an ellipse 
          defined as
          \[
                v_x^2/0.04 + v_y^2/0.0025  = 1.
          \]
          The elliptical selection is shown graphically on Fig. \ref{vertex_xy}.
\end{itemize}
%%%%%%%%%%%%%%
%   Figure   %
%%%%%%%%%%%%%%
\begin{figure}[t]
    \centering
    \includegraphics[width=.9\textwidth]{images/20160503_vertex_xy.png}
    \caption{Vertex position at the target.}
    \label{fig:vertex_xy}
\end{figure}  
%%%%%%%%%%%%%%
%   Figure   %
%%%%%%%%%%%%%%

\subsection{Radiative Selection}

As discussed in Chapter 2, the kinematic similarities between heavy photons and
radiative processes can be used to analyze both the rate of $A'$ signal production
and the sensitivity of the experiment to $A'$ signals.  It is then crucial to
maximize the fraction of radiative events in the final event sample.  The final
event sample is expected to be dominated by the Bethe-Heitler process.  However,
the kinematic difference between the Radiative and Bethe-Heitler processes can 
be exploited to reduce the number of Bethe-Heitler events.  Specifically, the 
$e^+e^-$ pair produced in a Radiative process will be highly boosted, while 
only one of the electrons in the Bethe-Heitler process will be boosted 
while the other will be much softer.  With this in mind, the sum of the momentum
of the electron and positron, ``p-sum'', allows discrimination between the two processes.
Specifically, radiative events are expected to have a ``p-sum'' peaked 
at the beam energy, while the distribution of Bethe-Heitlers will be peaked a 
low p-sum.

Figure \ref{} shows the sum of the momentum of the $e^+e^-$ tracks composing a 
pair. Using pure radiative and full trident MC, requiring a pair to have a 
p-sum cut greater than 0.8 GeV was found to maximize the radiative fraction.
The cut is shown graphically on Fig. \ref{}.  The invariant mass distribution
before and after the p-sum cut was applied is also shown on Fig. \ref{}.

\begin{figure}[t]
    \centering
    \includegraphics[width=1.0\textwidth]{images/invariant_mass_final.png}
    \caption{The Heavy Photon Search $e^+e^-$ invariant mass distribution after
             the final event selection has been applied.  The mass distribution
             will serve as the starting point for the resonance search.}
    \label{fig:mass_distribution}
\end{figure}

\subsection{Event Selection Efficiencies}

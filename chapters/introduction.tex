
\chapter{Introduction}

The Standard Model (SM) of particle physics continues to be one of mankind's 
greatest intellectual achievements. With the discovery of the Higgs boson at the
Large Hadron Collider in 2012 by both the ATLAS and CMS experiments
\cite{Aad:2012tfa, Chatrchyan:2012xdj}, all particles predicted by the SM have
been observed.  However, there remains many outstanding issues which the SM
fails to explain.  

One such issue is the composition and nature of dark matter (DM). The 
existence of DM was first infered in the early 1930's by Zwicky when calculating
the velocity dispersion of the galaxies in the Coma cluster 
\cite{Zwicky:1933gu}.  Using the velocity dispersions, Zwicky calculated the 
cluster's mass using the virial theorem and found it to be $\sim$ 400 times 
larger than what was expected from their luminosity.  He then concluded that 
the Coma cluster contained far more of some yet unobserved \emph{dunkle Materie}
or `dark matter' than luminous matter.  Additional evidence would come 
a few decades later when Rubin and Ford observed that the rotational velocity
of galxies was approximately flat instead of decreasing at $1/\sqrt{r}$ as 
expected \cite{Rubin:1980zd}.  More recent results based on gravitational
lensing \cite{Clowe:2006eq} and the cosmic microwave background 
\cite{Adam:2015rua}, further strengten the argument for the existence of dark 
matter.

In 2008, the Payload for Antimatter Matter Exploration and Light-nuclei 
Astrophysics (PAMELA) observed a rise in the positron fraction \cite{}.  This
lead to a surge of models where DM was framed to belogn to a ``hidden sector''
with its interactions mediated by a gauge boson associated with an additional
$U(1)$ gauge symmetry of nature.  The additional gauge boson  ($A'$, ``dark'',
``hidden'', ``heavy'' photon) couples to electric charge through ``kinetic 
mixing'' with the photon allowing a portal through which the properties of not
only dark matter but other hidden sector particles can be explored.

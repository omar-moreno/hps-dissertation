
\chapter{Resonance Search and Results}

\section{Overview}

If a heavy photon does indeed exist and has a mass that is within the acceptance
of the HPS detector, it will appear as a resonance above the copious QED trident
invariant mass spectrum.  Such a signal is expected to be Gaussian in shape,
centered at the $A'$ mass, have unknown normilization and a width equal to the
measured mass resolution of the experiment.

Searching for such a resonance is performed by scanning the measured
invariant mass distribution described in the previous chapter for any 
significant peaks.  Specifically, a window
is constructed around the $A'$ mass hypothesis and fit using a model composed
of a polynomial, for the background, and a Gaussian for the signal.  The width
and mean of the signal are taken as fixed parameters while the normalization of
the Gaussian and the background as well as the coefficients of the polynomials
are chosen such that they maximize the Poisson likelihood of the data.  

In order to assess if the composite model is significantly different from the 
background only model, a likelihood ratio test is performed.  The test yields a
value whose probability is distributed as a $\chi^2$ distribution with 1 degree
of freedom.  The $\chi^2$ value is used to calculate a p-value the quantifies 
the probability 
that the data is consistent with the background only hypothesis.  If the data 
is found to be consistent with the background-only model, an upper limit on the
number of signal events is set.  In turn, this is used to set a limit on the 
coupling strength of the $A'$.

The following chapter will discuss the details and results from a resonance 
search conducted on the 2015 engineering run data.  This includes a discussion
of the toy studies used to optimize the fitter along with the statistical 
formalism used to set a limit.


\section{Pseudo Data Sets}

Pseudo data sets are needed to study the fitter systematics and to optimize both
the fit function and window.  In order to obtain an invariant mass probability
density function (PDF) that describes the data, a smoothing algorithm was
% What smoothing algorithm was used? Why? Was a K-S test used to make sure the
% resulting PDF matched the data set?
applied to the normalized final invariant mass MC distribution.  The resulting
PDF after smoothing is shown on Fig.
% \ref{fig:smooth_pdf}. 

Events were sampled from the PDF between 0 and 100 MeV with the total number of
% What kind of sampling algorithm was used? 
events choosen to match the number observed in data.  The resulting pseudo data
is binned to match the data distributions with the expectation value of each 
bin Poisson distributed. 
